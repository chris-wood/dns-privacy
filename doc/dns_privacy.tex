% THIS IS SIGPROC-SP.TEX - VERSION 3.1
% WORKS WITH V3.2SP OF ACM_PROC_ARTICLE-SP.CLS
% APRIL 2009

\documentclass{llncs}

\usepackage{todonotes,times}
\usepackage{algorithm}
\usepackage{url}
\usepackage[noend]{algpseudocode}
\usepackage{mdframed}
\usepackage{amsmath}
\usepackage{paralist}
\usepackage{multirow,paralist}
\usepackage{fancybox}

\newenvironment{fminipage}%
{\begin{Sbox}\begin{minipage}}%
{\end{minipage}\end{Sbox}\fbox{\TheSbox}}

\makeatletter
%\renewcommand{\ALG@beginalgorithmic}{\scriptsize}
\makeatother

\newtheorem{defn}{\textbf{Definition}}
\newtheorem{thm}{\textbf{Theorem}}
\newtheorem{cor}{\textbf{Corollary}}
%\newtheorem{lemma}{\textbf{Lemma}}

\begin{document}

\mainmatter              % start of the contributions
\title{On the Efficacy of DNS Resolver Privacy Preservation}

\author{Gene Tsudik and Christopher A. Wood}

\institute{University of California Irvine, Irvine CA, USA\\
\email{\{gene.tsudik,woodc1@\}uci.edu}}

% typeset the title of the contribution
\maketitle

\begin{abstract}
TODO
\end{abstract}

%%% http://software.imdea.org/~bkoepf/papers/ndss13.pdf

\section{Introduction}
The need for a private Domain Name System (DNS) has become increasingly important
in recent years. There are several different proposals to address this
growing problem, including DNS-over-TLS and DNSCurve.
The former enables clients to create ephemeral sessions with either
their resolver or authoritative (stub) servers in which queries can be issued.
The latter uses per-query encryption to protect queries between clients and servers.
Encryption is core mechanism used to enable client privacy in both of these
solutions. However, in a recent study, Shulman showed that encryption alone
is insufficient to protect the privacy of queries. Information leaked
in DNS side channels, such query timing, frequency, and resolution ``chains,''
may reveal the contents of a query. Moreover, by observing the
trust properties of DNS servers and their responses, an adversary may also
learn the specific record within a domain that was requested.

There are a variety of mechanisms that can be used to plug these side channels, including:
message padding, query chaffing, query partitioning or splitting, and message interleaving.
Each of these techniques can increase the amount of entropy of a given query. Message buffering
can also be used to minimize information that is leaked through timing side channels.
Using query traces collected through DNS-OARC, we systematically study the efficacy of these
techniques against Shulman's attacks. We compare the privacy gains against the observed slowdown
induced by these privacy-preserving techniques.

Moreover, Shulman also showed that caching resolvers can be identified through timing side channel
attacks. We discuss several resolver techniques that can be used to deter these attacks
without introducing extra load on the authoritative name servers. Specifically, we study
randomized response delays to clients to mask the presence of caches. With
appropriately computed delays, resolver identification becomes difficult.

Finally, to complement query and resolver privacy, we also study client anonymity. In particular, we seek to
learn to what extent (cleartext) DNS query patterns can be linked to individual users.
Trivial linkability attacks mean that stub servers can learn information about individual
clients, even if encryption (without mutual authentication) is used to protect queries
in transit. Using both supervised and unsupervised machine learning algorithms,
we conducted linkability experiments in a scenario with only two users browsing the web
for a large amount of time (e.g., the course of an entire day). Our results indicate that
query patterns have no discernible impact on client anonymity.



The rest of this report is organized as follows. In Section \ref{sec:model} we formalize
the adversarial model and XXX

\section{System Model}\label{sec:model}
The DNS system is composed of clients $C$, recursive resolvers $R$, and stub (authoritative)
servers $S$. In the simplest use case, clients want to map a domain name to an IP address to
establish a connection with a web server or host. Clients express queries to a recursive
resolver that is responsible for finding the answer to this mapping query. If the answer
to the query has not yet been fetched and previously cached, then the recursive resolver
proceeds to ask the question to stub servers, starting at the root for the top-level-domain (TLD).
For example, if the client query is {\tt a.b.com}, then the stub associated with the com
TLD is queried for the answer. Among the possible options, the stub may return either an
address (in an A record) or a pointer to another stub server to query (in a NS record).
The resolver will recursively query stub servers until (a) an address is returned or (b)
a ``non-existent'' flag indicating that the name cannot be resolved to an address. This
final result is then relayed to the client to complete the process.

Traditionally, DNS packets are sent in cleartext over UDP between all parties. This has
recently been flagged as a privacy concern \cite{bortzmeyer2013}. Recently, a specification to run the
$C-R$ phase of the protocol over TLS to preserve query privacy (confidentiality) was
proposed \cite{dnstls}. In theory, this prevents an eavesdropping adversary between $C$ and $R$ from
learning the contents of a query. DNSCurve is an alternative scheme to individually encrypt
requests and responses to achieve the same degree of privacy \cite{dnscurve}. However,
Shulman showed that side channel information in the protocol, such as the timing and sizes
of requests and responses, reveal a non-negligible amount of information about the
queries \cite{shulman}.

XXX: formal modeling to plug leakage... outline our experiment and plan, and then discuss why it wasn't so great





\medskip
\small
\bibliographystyle{abbrv}
\bibliography{ref}

\end{document}
